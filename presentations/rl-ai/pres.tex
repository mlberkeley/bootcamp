%NEURAL PRESY ;)

%----------------------------------------------------------------------------------------
%   PACKAGES AND THEMES
%----------------------------------------------------------------------------------------

\documentclass{beamer}

\mode<presentation> {

\usetheme{Berkeley}
%\usetheme{Berlin}


\setbeamertemplate{navigation symbols}{} % To remove the navigation symbols from the bottom of all slides uncomment this line
}

\usepackage{graphicx} % Allows including images
\usepackage{booktabs} % Allows the use of \toprule, \midrule and \bottomrule in tables
\usepackage{caption}
\usepackage{subcaption}
\usepackage{algorithm,algorithmic}

% tikz and associated macros
\usepackage{tikz}
\usepackage{tikz-cd}

\usepackage{pgfplots}
\def\layersep{2cm}
\def\nodesep{0.25cm}
\newcommand*{\Scale}[2][4]{\scalebox{#1}{$#2$}}%
\newcommand*{\Resize}[2]{\resizebox{#1}{!}{$#2$}}%
\newcommand\sep{1.9cm}
\newcommand\height{0.9cm}
\usetikzlibrary{decorations.pathmorphing, backgrounds}
\tikzset{snake it/.style={decorate, decoration=snake}}

%
%

% math
\usepackage{amsthm}
\usepackage{amsmath}
\usepackage{amssymb}
\usepackage{mathabx}

\numberwithin{equation}{subsection}
\numberwithin{theorem}{subsection}

\DeclareSymbolFont{cmlargesymbols}{OMX}{cmex}{m}{n}
\let\sumop\relax
\DeclareMathSymbol{\sumop}{\mathop}{cmlargesymbols}{"50}


\def\reals{{\mathbb R}}
\def\torus{{\mathbb T}}
\def\integers{{\mathbb Z}}
\def\rationals{{\mathbb Q}}
\def\expect{\mathop{{\mathbb{E}}}}
\def\tens{\mathop{{\bigotimes}}}
\def\naturals{{\mathbb N}}
\def\complex{{\mathbb C}\/}
\def\distance{\operatorname{distance}\,}
\def\support{\operatorname{support}\,}
\def\dist{\operatorname{dist}\,}
\def\Span{\operatorname{span}\,}
\def\degree{\operatorname{degree}\,}
\def\kernel{\operatorname{kernel}\,}
\def\dim{\operatorname{dim}\,}
\def\codim{\operatorname{codim}}
\def\trace{\operatorname{trace\,}}
\def\dimension{\operatorname{dimension}\,}
\def\codimension{\operatorname{codimension}\,}
\def\kernel{\operatorname{Ker}}
\def\Re{\operatorname{Re\,} }
\def\Im{\operatorname{Im\,} }
\def\eps{\varepsilon}
\def\lt{L^2}
\def\bull{$\bullet$\ }
\def\det{\operatorname{det}}
\def\Det{\operatorname{Det}}
\def\diameter{\operatorname{diameter}}
\def\symdif{\,\Delta\,}
\newcommand{\norm}[1]{ \|  #1 \|}
\newcommand{\set}[1]{ \left\{ #1 \right\} }
\def\suchthat{\mathrel{}\middle|\mathrel{}}
\def\one{{\mathbf 1}}
\def\cl{\text{cl}}

\def\newbull{\medskip\noindent $\bullet$\ }
\def\nobull{\noindent$\bullet$\ }
\def\defeq{\stackrel{\text{def}}{=}}


\def\scriptf{{\mathcal F}}
\def\scriptq{{\mathcal Q}}
\def\scriptg{{\mathcal G}}
\def\scriptm{{\mathcal M}}
\def\scriptb{{\mathcal B}}
\def\scriptc{{\mathcal C}}
\def\scriptt{{\mathcal T}}
\def\scripti{{\mathcal I}}
\def\scripte{{\mathcal E}}
\def\scriptv{{\mathcal V}}
\def\scriptw{{\mathcal W}}
\def\scriptu{{\mathcal U}}
\def\scriptS{{\mathcal S}}
\def\scripta{{\mathcal A}}
\def\scriptr{{\mathcal R}}
\def\scripto{{\mathcal O}}
\def\scripth{{\mathcal H}}
\def\scriptd{{\mathcal D}}
\def\scriptl{{\mathcal L}}
\def\scriptn{{\mathcal N}}
\def\scriptp{{\mathcal P}}
\def\scriptk{{\mathcal K}}
\def\scriptP{{\mathcal P}}
\def\scriptj{{\mathcal J}}
\def\scriptz{{\mathcal Z}}
\def\scripts{{\mathcal S}}
\def\scriptx{{\mathcal X}}
\def\scripty{{\mathcal Y}}
\def\frakv{{\mathfrak V}}
\def\frakG{{\mathfrak G}}
\def\frakB{{\mathfrak B}}
\def\frakC{{\mathfrak C}}



%----------------------------------------------------------------------------------------
%   TITLE PAGE
%----------------------------------------------------------------------------------------

\title[ML@B Bootcamp]{Introduction to Reinforcement Learning and AI}

\author[James Bartlett \& Will Guss]{Will Guss \\ James Bartlett}
\date{} % Date, can be changed to a custom date
\makeatletter
\newcommand{\verbatimfont}[1]{\renewcommand{\verbatim@font}{\ttfamily#1}}
\makeatother
\begin{document}

\begin{frame}
\titlepage
\end{frame}

\addtobeamertemplate{frametitle}{}{%
\begin{tikzpicture}[remember picture,overlay]
\node[anchor=north east,yshift=10pt] at (current page.north east) {\includegraphics[height=2cm]{White_outline_small_name_transparent.png}};
\end{tikzpicture}}

\begin{frame}

\begin{center}
\Huge Agenda 
\end{center}


\frametitle{Overview}
\tableofcontents
\end{frame}


%----------------------------------------------------------------------------------------
%   PRESENTATION SLIDES
%----------------------------------------------------------------------------------------
\section{Background}
\begin{frame}
  \frametitle{Markov Decision Process (MDP)}
    Environment, $E = (\mathcal{S}, \mathcal{A}, \mathcal{R}, \rho, r)$. 
    \begin{enumerate}
    \item State space, $\mathcal{S} = \mathbb{R}^n$
    \item Action space, $\mathcal{A} = \mathbb{R}^m$
    \item Reward space, $\mathcal{R} = \mathbb{R}$   
    \item Transition function, $\rho(s'\ |\ s,a)$. Given a previous state $s$ and action $a$, environment gives $s'$.
    \item Reward function $r(s,a) \in \mathcal{R}$.
    \end{enumerate}
\end{frame} 
\begin{frame}[fragile]
  \frametitle{Reinforcement Learning Agent}
    Deterministic agent $\pi: \mathcal{S} \to \mathcal{A}$ acts in $E$. 
    \begin{equation*}
      \begin{tikzcd}
          s_1 \arrow{r}{\pi} & a_1 \arrow{r}{\rho, r} & s_2, r_2 \arrow{r}{\pi}& a_2  \arrow{r}{\rho, r} & \cdots
         \end{tikzcd}   
    \end{equation*}
    Eg. Pacman sees the screen, and decides to move $\uparrow, \downarrow, \rightarrow, \leftarrow$ and then gets a reward for eating food.
\end{frame}
\begin{frame}
\frametitle{Reinforcement Learning Agent}
  \begin{center}
    \includegraphics[width=0.5\textwidth]{Pac-man.png}
  \end{center}
\end{frame}

\begin{frame}
\frametitle{Value in Reinforcement Learning}
  The value of a given state for an agent $\pi$ is defined as 
  \begin{equation*}
    V^\pi(s_t) = \sum_{n={t+1}}^\infty \gamma^n r(s_n, \pi(s_n))
  \end{equation*} 
  \begin{enumerate}
    \item $\gamma$ is the discount factor
    \item $\pi(s_n)$ is the action the agent $\pi$ makes after seeing state $s_n$.
    \item $r(s_n, \pi(s_n))$ is the reward the agent gets from taking that action.
  \end{enumerate}
\end{frame}

\begin{frame}
\frametitle{Expected Future Reward}
  The expected future reward of an agent $\pi$, also known as the Q function, is
  \begin{equation*}
    Q^\pi(s_t, a_t) = \underbrace{r(s_t, a_t)}_{\text{reward for } a_t} + V^\pi(s_t)
  \end{equation*}
\end{frame}

\begin{frame}
\frametitle{Bellman Equation}
  The Bellman equation says 
  \begin{equation*}
    Q^\pi(s_t, a_t) = r_t + \gamma Q^\pi(s_{t+1}, \pi(s_{t+1}))
  \end{equation*}
  Given some state $s_t$, the \textbf{best} agent, $\pi^*$ is one that take action 
  \begin{equation*}
    a_t = \arg \max_a Q(s_t, a).   
  \end{equation*}
\end{frame}

\section{Algorithms}
\section{Questions}
\begin{frame}
\Huge{\centerline{Questions?}}
\end{frame}

%----------------------------------------------------------------------------------------

\end{document}
